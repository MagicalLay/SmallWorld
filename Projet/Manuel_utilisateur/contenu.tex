\section{Principes et but du jeu}

Il s’agit d’un jeu au tour-par-tour inspiré de \og SmallWorld \fg{} dans lequel chaque joueur dirige un peuple. Le but du jeu est de gérer des unités sur une carte pour obtenir le plus de points possible, à la fin d’un certain nombre de tours. Pour ce faire, chaque joueur commence avec ses unités placées sur une même case de la carte. Il doit ensuite les répartir au mieux sur les différentes cases de la carte : par défaut, une case rapporte 1 point. Les unités d’un joueur peuvent également attaquer les unités de l'autre pour les détruire (limitant ainsi l’acquisition de points de l’adversaire) et occuper une case de la carte. Les points sont recomptés à la fin de chaque tour pour que tous les joueurs connaissent leur nombre courant de points ainsi que celui de leurs adversaires.

\section{Rappel des règles du jeu}

\subsection{Description des peuples}

Il existe trois peuples Elfe (\textit{Elf}), Orc (\textit{Orc}), et Nain (\textit{Dwarf}), ayant des caractéristiques très différentes :

\begin{itemize}

\item \textbf{Elf} :  le coût de déplacement sur une case Forêt est divisé par deux. Le coût de déplacement sur une case Désert est multiplié par deux.
\item \textbf{Orc} : le coût de déplacement sur une case Plaine est divisé par deux. Une unité Orc n’acquière aucun point sur les cases de type Forêt. Lorsqu’une unité Orc détruit une autre unité, elle possède alors 1 point de bonus permanent. Cet effet est cumulable et est lié à chaque unité (i.e. si l’unité ayant le bonus meurt, le bonus disparaît).
\item \textbf{Nain} :  le coût de déplacement sur une case Plaine est divisé par deux. Une unité Nain n’acquière aucun point sur les cases Plaine. Lorsqu’elle se trouve sur une case Montagne, une unité Nain a la capacité de se déplacer sur n’importe quelle case Montagne de la carte à condition qu’elle ne soit pas occupée par une unité adverse.

\end{itemize}

À chaque tour, toutes les unités peuvent être déplacées et attaquer, mais il est aussi possible de passer son tour. Par défaut, chaque unité possède deux points de mouvement et chaque déplacement en coûte un (hors bonus/malus). Cela signifie qu’une même unité peut se déplacer ou attaquer plusieurs fois par tour. Chaque unité possède 2 points d’attaque, 1 point de défense et 5 points de vie. Les unités ne récupèrent pas leurs points de vie à la fin d’un tour, mais les points de déplacement sont réinitialisés à 2.

\subsection{La carte du monde}

La carte du monde se compose de cases hexagonales. Il existe quatre types de case : Plaine (\textit{Field}), Désert (\textit{Desert}), Montagne (\textit{Mountain}), Forêt (\textit{Forest}). 

Il existe 3 tailles de carte :

\begin{itemize}

\item \textbf{Petite} (\textit{Small}) : 6 cases * 6 cases, 5 tours, 4 unités par peuple.
\item \textbf{Moyenne} (\textit{Medium}) : 10 cases * 10 cases, 20 tours, 6 unités par peuple.
\item \textbf{Grande} (\textit{Large}) : 14 cases * 14 cases, 30 tours, 8 unités par peuple.

\end{itemize}

\subsection{Aspect du jeu}

La carte, ses ressources, les unités de tous les peuples sont visibles par tous les joueurs. La carte est vue du dessus.

\section{Utilisation de l'interface}

\subsection{Lancement du jeu}

\subsection{Créer une partie}

Pour créer une partie à partir de l'écran d'accueil, vous devez cliquer sur \og Create New Game \fg{}. Une boîte de dialogue s'affiche alors, par laquelle il faut choisir la taille de la carte par sélection dans la liste déroulante. Après validation, une nouvelle fenêtre s'affiche et chaque joueur doit choisir son peuple (une vérification a été mise en place afin de s'assurer que les peuples choisis soient différents). Une fenêtre s'affiche alors pour vous demander confirmation de vos choix : cliquez sur OK pour commencer la partie. 


Chaque peuple débute la partie avec toutes ses unités sur la même case de la carte, choisie de manière à ce que les joueurs ne soient pas trop proches. L’ordre de jeu est choisi aléatoirement en début de partie. Les joueurs jouent chacun leur tour sur un même ordinateur. Deux joueurs ne peuvent sélectionner le même peuple.